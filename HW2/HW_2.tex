%!TEX program = xelatex
%# -*- coding: utf-8 -*-
%!TEX encoding = UTF-8 Unicode

\documentclass[12pt,oneside,a4paper]{article}
\usepackage{geometry}
\geometry{verbose,tmargin=2cm,bmargin=2cm,lmargin=2cm,rmargin=2cm}
\usepackage[pdfusetitle,
 bookmarks=true,bookmarksnumbered=true,bookmarksopen=true,bookmarksopenlevel=2,
 breaklinks=false,pdfborder={0 0 1},backref=false,colorlinks=false]
 {hyperref}
\hypersetup{pdfstartview={XYZ null null 1}}
\usepackage{url}
\setcounter{secnumdepth}{2}
\setcounter{tocdepth}{2}
\usepackage{microtype}

\usepackage{amsmath, amsthm, amssymb, amsfonts}
\usepackage[retainorgcmds]{IEEEtrantools}

\usepackage{algorithm}
\usepackage{algorithmic}
\renewcommand{\algorithmicrequire}{\textbf{Input:}} 
\renewcommand{\algorithmicensure}{\textbf{Output:}} 

\usepackage[sc]{mathpazo}
\linespread{1.1}
\usepackage[T1]{fontenc}


\usepackage{graphics}
\usepackage{graphicx}
\usepackage[figure]{hypcap}
\usepackage[hypcap]{caption}
\usepackage{tikz}
%\usepackage{grffile} 
%\usepackage{float} 
\usepackage{pdfpages}

\usepackage{multirow}
\usepackage{booktabs}
\usepackage{threeparttable}

%\usepackage[square,numbers,super,comma,sort]{natbib}
%\usepackage[backend=biber, style=nature, sorting=none, isbn=false, url=false, doi=false]{biblatex}
%\addbibresource{ref.bib}
%\usepackage[]{authblk}

\usepackage{verbatim}
\usepackage{listings}

\newcommand{\problem}[1]
{
    \clearpage
    \section*{Problem {#1}}
}

\newcommand{\subproblem}[1]
{
    \subsection*{Problem {#1}}
}


\newcommand{\solution}
{
    \vspace{15pt}
    \noindent\ignorespaces\textbf{\large Solution}\par
}

\usepackage{fancyhdr}
\usepackage{extramarks}
\lhead{\hmwkAuthorName}
\chead{\hmwkTitle}
\rhead{\firstxmark}
\cfoot{\thepage}

\newcommand{\hmwkTitle}{CSCI 5304 HW 2}
\newcommand{\hmwkAuthorName}{Jingxiang Li}

\setlength\headheight{15pt}
\setlength\parindent{0pt}
\setlength{\parskip}{0.5em}

\newcommand{\m}[1]{\texttt{{#1}}}


\pagestyle{fancy}

\title{\hmwkTitle}
\author{\hmwkAuthorName}
\date{\today}

\begin{document}
\problem{1}
Consider the two mathematically equivalent formulas
\begin{enumerate}
    \item $a = x^2 - y^2$
    \item $a = (x + y) * (x - y)$
\end{enumerate}

\subproblem{a}
Compute both formulas in 2 digit decimal arithmetic using $x = 11$, $y = 10$. Which formula gives the best answer? Ignore any limits on the exponent.

\solution
$x = .11\times 10^2$, $y = .10\times 10^2$
\begin{enumerate}
    \item $x ^ 2 = .0121 \times 10^4 = .12 \times 10^3$, $y^2 = .0100 \times 10^4 = .10 \times 10^3$, \\
    $a = x^2 - y^2 = .02 \times 10^3 = .20 \times 10^2$ 
    \item $x + y = .21 \times 10^2$, $x - y = .01 \times 10^2 = .10 \times 10^1$, \\
    $a = (x+y)\times(x-y) = .21 \times 10^2 * .10 \times 10^1 = .21 \times 10^2$
\end{enumerate}
Note that formula 2 gives the best answer.

\subproblem{b}
Compute both formulas in 3-bit binary arithmetic using $x = 3/2$ (dec) and $y = 1$. Again, which formula gives the most accurate answer? Ignore any limits on the exponent.

\solution
$x = .110 \times 2^1$, $y = .100 \times 2^1$
\begin{enumerate}
    \item $x^2 = .100100 \times 2^2 = .100 \times 2^2$, $y^2 = .010000 \times 2^2 = .100 \times 2^1$, \\
    $a = x^2 - y^2 = .100 \times 2^2 - .0100 \times 2^2 = .010 \times 2^2 = .100 \times 2^1$
    \item $x + y = 1.010 \times 2^1 = .101 \times 2^2$, $x - y = .010 \times 2^1 = .100 \times 2^0$, \\
    $a = (x + y) \times (x - y) = .010100 \times 2^2 = .101 \times 2^1$
\end{enumerate}
Note that formula 2 gives the best answer.

\subproblem{c}
In your favorite programming language and platform, find two numbers $x$, $y$ such that the two formulas above give different answers. Which answer is more accurate. Report the precision used, the machine epsilon (unit round-off) and the general description of your machine. Can you find two numbers for which one of the answers has no accuracy whatsoever, but the other is almost OK? Note: in Matlab all arithmetic is in double precision, but you can force single precision by using the single function: $\m{a = single(5/4)}$ forces all arithmetic involving a to be on single precision.

\solution

Progarmming Language: Octave 64-bit\\
Precision: Double Precision\\
epsilon: $2^{-52}$\\
Machine: Intel Core i5-3210M CPU @ 2.50GHz$\times$4, 64-bit\par

Let $x = \frac{1}{2} + \mathrm{eps}$, $y = \frac{1}{2}$, and the second fomula has more accuracy.

\begin{verbatim}
x = 1/2 + eps;
y = 1/2;
format hex;
a1 = x^2 - y^2;
a2 = (x + y) * (x - y);
a1
> a1 = 3cb0000000000000
a2
> a2 = 3cb0000000000001
\end{verbatim}

\problem{2}
Consider the system
$$Ax \equiv \begin{pmatrix}
  -0.001 & 1.001\\
  0.001  & -0.001\\
 \end{pmatrix} \begin{pmatrix}
  x_{1}\\
  x_{2}\\
 \end{pmatrix} = \begin{pmatrix}
  1\\
  0\\
 \end{pmatrix} \equiv b$$
whose solution is $x_{1} = x_{2} = 1$ and the system 
$$(A + \Delta A)y = b + \Delta b$$
where $\Delta A = \epsilon |A|$, $\Delta b = \epsilon |b|$. Here $|A|$ means take the absolute value of all the elements individually. In the following we let $\epsilon = 10^{-4}$.

\subproblem{a}
Compute $k_{\infty}(A)$. Compute the actual value of $||x-y||_{\infty} / ||x||_{\infty}$ and its estimate obtained from using the (standard) condition number $k_{\infty}(A)$ (Theorem 2 in notes).\par
Which is quite far from the actual error.

\solution
At first we calculate the actual value using Octave.

\begin{verbatim}
A = [-0.001, 1.001; 0.001, -0.001];
b = [1; 0];
epsilon = 1e-04;

A_new = A + epsilon .* abs(A);
b_new = b + epsilon .* abs(b);
y = inv(A_new) * b_new;
x = [1; 1];
norm(x - y, inf) / norm(x, inf)
> 2.0038e-04
\end{verbatim}

As the result, the actual error is $2.0038\mathrm{e}-04$

Then we calculate its estimate using the following formula
$$\frac{||x-y||_{\infty}}{||x||_{\infty}} \leq \frac{2\epsilon}{1 - \rho}|||A^{-1}||A|||_{\infty}$$
where $\rho = \epsilon k_{\infty}(A)$

\begin{verbatim}
rho = epsilon * cond(A, inf);
2 * epsilon / (1 - rho) * norm(abs(A) * abs(inv(A)), inf)
> 6.6785e-04
\end{verbatim}

Note that the estimation of error bound is $6.6785\mathrm{e}-04$

\subproblem{b}
Now repeat the above, but this time use $\Delta(b) = \epsilon \begin{pmatrix}
0\\
1
\end{pmatrix}$

\solution
At first we calculate the actual value using Octave.

\begin{verbatim}
A = [-0.001, 1.001; 0.001, -0.001];
b = [1; 0];
epsilon = 1e-04;

A_new = A + epsilon .* abs(A);
b_new = b + epsilon .* abs([0; 1]);
y = inv(A_new) * b_new;
x = [1; 1];
norm(x - y, inf) / norm(x, inf)
> 0.099790
\end{verbatim}

As the result, the actual error is $0.099790$ \par

Then we calculate its estimate

\begin{verbatim}
norm(inv(A), inf) * norm(A, inf) / ...
	(1 - norm(inv(A), inf) * norm(epsilon * abs(A), inf)) *...
	(norm(epsilon * [0;1], inf) / norm(b, inf) + norm(epsilon * abs(A), inf) /...
	norm(A, inf))
> 0.22321
\end{verbatim}

Note that the estimation of error bound is $0.22321$

\problem{3}
\subproblem{a}
Show that the following matrix is singular
$$A = \begin{pmatrix}
1&2&-1\\
2&1&1\\
-1&1&-2
\end{pmatrix}$$

\solution
Let $A_{i}$ be the $i$th column of matrix $A$, then we have $A_{1} - A_{2} - A_{3} = 0$, suggesting $\mathrm{rank}(A) = 2 < 3$, so that $A$ is singular.

\subproblem{b}
What is the range or column space of $A$ ? What is its null space? Give a basis for each subspace.

\solution
Since $\mathrm{rank}(A) = 2$, the column space of $A$ should be the linear combination of any two columns of $A$, which is $\{x: x = \alpha A_{1} + \beta A_{2}, \alpha,\beta \in \mathbb{R} \}$, where $A_{1} = \begin{pmatrix}
1\\
2\\
-1
\end{pmatrix}$, $A_{2} = \begin{pmatrix}
2\\
1\\
1
\end{pmatrix}$.
The basis of the column space of $A$ can be $\{A_{1}, A_{2}\}$

Then, the null space will be $\{x: x = \alpha\begin{pmatrix}
-1\\
1\\
1
\end{pmatrix}, \alpha \in \mathbb{R}\}$
the basis of the null space of $A$ can be $\{\begin{pmatrix}
-1\\
1\\
1
\end{pmatrix}\}$

\subproblem{c}
Consider the matrix $B$ obtained from $A$ by adding $\eta = 0.001$ to the entry (1, 3) (So $B = A + \eta e_{1} e^{T}_{3}$). Without computing the inverse of $B$, show that $||B^{-1}||_{1} \geq 3,000$.

\solution
Let $x = \begin{pmatrix}
-1\\
1\\
1
\end{pmatrix}$, then we have $Bx = \begin{pmatrix}
  -0.001\\
   0\\
   0
\end{pmatrix}$, then we can have the lower bound of 
$$||B^{-1}||_{1} \geq \frac{||x||_{1}}{||Bx||_{1}} = 3,000$$
Q.E.D.

\subproblem{d}
Find a lower bound for the condition number $k_{1}(B)$.

\solution
$$k_{1}(B) = ||B||_{1} \cdot ||B^{-1}||_{1} \geq 4 \times 3,000 = 12,000$$

\problem{4}
Consider the $n\times n$ matrix
$$A_{n} = \begin{pmatrix}
1 &  & & & & &\\
-2& 1& & & & &\\
  &  -2& 1&  & & &\\
  & &\ddots&\ddots& & &\\
  &  & & \ddots & \ddots & &\\
    & &  & & -2 & 1
\end{pmatrix}$$
What is $A_{n}^{-1}$? [Hint: Write $A_{n} = I - E_{n}$ and use expansion $I + E + E^{2} + \cdots$.]\par
Calculate the condition numbers $k_{1}(A_{n})$ and $k_{\infty}(A_{n})$. Verify your results with matlab for the case $n = 10$.

\solution
Note that we can rewrite $A_{n} = I - E$ where 
$$E = \begin{pmatrix}
0 &  & & & & &\\
2& 0& & & & &\\
  &  2& 0&  & & &\\
  & &\ddots&\ddots& & &\\
  &  & & \ddots & \ddots & &\\
    & &  & & 2 & 0
\end{pmatrix}$$
Then we have $(E)^n = 0$ and 
$$(I - E)\cdot(I + E + E^2 + \dots + E^{n-1}) = I - E^{n} = I$$
So that 
$$A_{n}^{-1} = (I + E + E^2 + \dots + E^{n-1})$$

Then we verify the result using Octave for the case $n = 10$
\begin{verbatim}
n = 10;
A = eye(n);
for i = 1 : n
    for j = 1 : n
        if (i - j == 1)
            A(i, j) = -2;
        end
    end
end
E = -(A - eye(n));

result = zeros(n);
for i = 0 : (n - 1)
    result = result + E ^ i;
end

%% A_inv is the inverse matrix of A obtained from formula above

A_inv = result
>
     1     0     0     0     0     0     0     0     0     0
     2     1     0     0     0     0     0     0     0     0
     4     2     1     0     0     0     0     0     0     0
     8     4     2     1     0     0     0     0     0     0
    16     8     4     2     1     0     0     0     0     0
    32    16     8     4     2     1     0     0     0     0
    64    32    16     8     4     2     1     0     0     0
   128    64    32    16     8     4     2     1     0     0
   256   128    64    32    16     8     4     2     1     0
   512   256   128    64    32    16     8     4     2     1

%% Verify A_inv is the inverse of A. 
%% If so, A * A_inv should be the identity matrix.

A * A_inv
>
     1   0   0   0   0   0   0   0   0   0
     0   1   0   0   0   0   0   0   0   0
     0   0   1   0   0   0   0   0   0   0
     0   0   0   1   0   0   0   0   0   0
     0   0   0   0   1   0   0   0   0   0
     0   0   0   0   0   1   0   0   0   0
     0   0   0   0   0   0   1   0   0   0
     0   0   0   0   0   0   0   1   0   0
     0   0   0   0   0   0   0   0   1   0
     0   0   0   0   0   0   0   0   0   1
\end{verbatim}

\end{document}