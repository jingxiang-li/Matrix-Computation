%!TEX program = xelatex
%# -*- coding: utf-8 -*-
%!TEX encoding = UTF-8 Unicode

\documentclass[12pt,oneside,a4paper]{article}
\usepackage{geometry}
\geometry{verbose,tmargin=2cm,bmargin=2cm,lmargin=2cm,rmargin=2cm}
\usepackage[pdfusetitle,
 bookmarks=true,bookmarksnumbered=true,bookmarksopen=true,bookmarksopenlevel=2,
 breaklinks=false,pdfborder={0 0 1},backref=false,colorlinks=false]
 {hyperref}
\hypersetup{pdfstartview={XYZ null null 1}}
\usepackage{url}
\setcounter{secnumdepth}{2}
\setcounter{tocdepth}{2}
\usepackage{microtype}

\usepackage{amsmath, amsthm, amssymb, amsfonts}
\usepackage[retainorgcmds]{IEEEtrantools}

\usepackage{algorithm}
\usepackage{algorithmic}
\renewcommand{\algorithmicrequire}{\textbf{Input:}} 
\renewcommand{\algorithmicensure}{\textbf{Output:}} 

\usepackage[sc]{mathpazo}
\linespread{1.1}
\usepackage[T1]{fontenc}
%\usepackage{garamondx}
%\usepackage[garamondx,cmbraces]{newtxmath}

\usepackage{graphics}
\usepackage{graphicx}
\usepackage[figure]{hypcap}
\usepackage[hypcap]{caption}
\usepackage{tikz}
%\usepackage{grffile} 
%\usepackage{float} 
\usepackage{pdfpages}

\usepackage{multirow}
\usepackage{booktabs}
\usepackage{threeparttable}

%\usepackage[square,numbers,super,comma,sort]{natbib}
%\usepackage[backend=biber, style=nature, sorting=none, isbn=false, url=false, doi=false]{biblatex}
%\addbibresource{ref.bib}
%\usepackage[]{authblk}

\usepackage{verbatim}
\usepackage{listings}
\usepackage{color}

\newcommand{\problem}[1]
{
    \clearpage
    \section*{Problem {#1}}
}

\newcommand{\subproblem}[1]
{
    \subsection*{Problem {#1}}
}


\newcommand{\solution}
{
    \vspace{15pt}
    \noindent\ignorespaces\textbf{\large Solution}\par
}

\renewcommand{\proof}
{
    \vspace{15pt}
    \noindent\ignorespaces\textbf{\large Proof}\par
}

\usepackage{fancyhdr}
\usepackage{extramarks}
\lhead{\hmwkAuthorName}
\chead{\hmwkTitle}
\rhead{\firstxmark}
\cfoot{\thepage}

\newcommand{\hmwkTitle}{CSCI 5304 HW 6}
\newcommand{\hmwkAuthorName}{Jingxiang Li}

\setlength\headheight{15pt}
\setlength\parindent{0pt}
\setlength{\parskip}{0.5em}

\newcommand{\m}[1]{\texttt{{#1}}}


\pagestyle{fancy}

\title{\hmwkTitle}
\author{\hmwkAuthorName}
\date{\today}

\begin{document}

\definecolor{mygreen}{rgb}{0,0.6,0}
\definecolor{mygray}{rgb}{0.5,0.5,0.5}
\definecolor{mymauve}{rgb}{0.58,0,0.82}

\lstset{ %
  backgroundcolor=\color{white},   % choose the background color; you must add \usepackage{color} or \usepackage{xcolor}
  basicstyle=\small\ttfamily,        % the size of the fonts that are used for the code
  breakatwhitespace=false,         % sets if automatic breaks should only happen at whitespace
  breaklines=true,                 % sets automatic line breaking
  captionpos=b,                    % sets the caption-position to bottom
  commentstyle=\color{mygreen},    % comment style
  deletekeywords={...},            % if you want to delete keywords from the given language
  escapeinside={\%*}{*)},          % if you want to add LaTeX within your code
  extendedchars=true,              % lets you use non-ASCII characters; for 8-bits encodings only, does not work with UTF-8
  frame=single,                    % adds a frame around the code
  keepspaces=true,                 % keeps spaces in text, useful for keeping indentation of code (possibly needs columns=flexible)
  keywordstyle=\color{blue},       % keyword style
  language=Octave,                 % the language of the code
  morekeywords={*,...},            % if you want to add more keywords to the set
  numbers=left,                    % where to put the line-numbers; possible values are (none, left, right)
  numbersep=10pt,                   % how far the line-numbers are from the code
  numberstyle=\tiny\color{mygray}, % the style that is used for the line-numbers
  rulecolor=\color{black},         % if not set, the frame-color may be changed on line-breaks within not-black text (e.g. comments (green here))
  showspaces=false,                % show spaces everywhere adding particular underscores; it overrides 'showstringspaces'
  showstringspaces=false,          % underline spaces within strings only
  showtabs=false,                  % show tabs within strings adding particular underscores
  stepnumber=1,                    % the step between two line-numbers. If it's 1, each line will be numbered
  stringstyle=\color{mymauve},     % string literal style
  tabsize=2,                       % sets default tabsize to 2 spaces
  title=\lstname,                   % show the filename of files included with \lstinputlisting; also try caption instead of title
  aboveskip=\baselineskip, 
  belowskip=-1 \baselineskip
}


\maketitle

\problem{1}
Let 
$A = \begin{bmatrix}
1 & 1 & 1\\
0 & 1 & 1\\
s & 0 & 2
\end{bmatrix}$, with $s = -10^{-6}$

\subproblem{a}
Compute all the eigenvalues $\lambda_{i}$ and corresponding right eigenvectors $x_{i}$ and left eigenvectors $y_{i}$.

\solution
$$\Lambda = \mathrm{diag}(\lambda_{1}, \lambda_{2}, \lambda_{3}) = \begin{bmatrix}
2.0000 & 0 & 0\\
0 & 1.0010 & 0\\
0 & 0 & 0.9990
\end{bmatrix}$$
$$X = (x_{1}, x_{2}, x_{3}) = \begin{bmatrix}
-0.8165 & 1.0000 & -1.0000 \\
-0.4082 & 0.0010 &  0.0010 \\
-0.4082 & 0.0000 & -0.0000
\end{bmatrix}$$
$$Y = (y_{1}, y_{2}, y_{3}) = \begin{bmatrix}
 0.0000 &  0.0007 & -0.0007\\
 0.0000 &  0.7064 &  0.7078\\
-1.0000 & -0.7078 & -0.7064
\end{bmatrix}
$$

\subproblem{b}
Compute the ``overall condition number'' for this eigenproblem based on the perturbation formula
$$|\lambda_{A + E} - \lambda_{A}| \leq ||X|| \cdot ||X^{-1}|| \cdot ||E||$$
where $X$ is the matrix of right eigenvectors and $E$ is some generic perturbation matrix.

\solution
$\mathrm{cond}_{0} = ||X|| \cdot ||X^{-1}|| = 1.6608 \times 10^{3}$

\subproblem{c}
Compute the condition numbers for every individual eigenvalue based on the perturbation approximation (ignoring higher order terms)
$$|\lambda_{A + E} - \lambda_{A}| \approx \frac{O(||E||)}{\cos{\theta}} = \frac{||y_{i}||_2||x_{i}||_2}{y_{i}^{T}x_{i}}O(||E||)$$
where $y_i$, $x_i$ are the left and right eigenvectors corresponding to the eigenvalue $\lambda_{A}$ for the matrix $A$. Do this for each eigenvalue of $A$..

\solution
$\mathrm{cond}_{1} = (2.4495, 707.8165, 706.4023)^T$

\subproblem{d}
Compute the eigenvalues of the matrix $\tilde{A}$ defined as the same as the matrix $A$ above but with $s = 0$. How do the eigenvalues of $\tilde{A}$ compare with what would be expected based on the condition number bounds?

\solution
Let's define $E = \begin{bmatrix}
0 & 0 & 0\\
0 & 0 & 0\\
-s & 0 & 0
\end{bmatrix}$, then we can apply the above result to guess the eigenvalues of $\tilde{A}$. Note that $||E||_{2} = -s$.

First consider the ``overall condition number'', it suggests that for each eigenvalue $\lambda_{A + E}$ of $A + E$ there exists an eigenvalue $\lambda_{A}$ of A such that  
$$|\lambda_{A + E} - \lambda_{A}| \leq ||X|| \cdot ||X^{-1}|| \cdot ||E|| = -s * 1.6608 \times 10^{3} = 1.6608 \times 10^{-3}$$

Then consider the ``condition numbers for every individual eigenvalue'', we know that 
$$
\begin{aligned}
|\lambda_{A+E}^{(1)} - \lambda_{A}^{(1)}| &\approx -s * 2.4495 = 2.4495 * 10^{-6}\\
|\lambda_{A+E}^{(2)} - \lambda_{A}^{(2)}| &\approx -s * 707.8165 = 7.0782 * 10^{-4}\\
|\lambda_{A+E}^{(3)} - \lambda_{A}^{(3)}| &\approx -s * 706.4023 = 7.0640 * 10^{-4}
\end{aligned}
$$

Note that the above two ``estimations'' are consistent with each other.

\problem{2}
Compute all the eigenvalues and eigenvectors of the complex symmetric matrix
$$A = \begin{bmatrix}
2i & 1\\
1 & 0
\end{bmatrix}$$

\solution
$\lambda_{1} + \lambda_{2} = 2i$, $\lambda_{1} \cdot \lambda_{2} = -1$, thus $\lambda_{1} = \lambda_{2} = i$. Then we consider the null space of $A - i \cdot I$, thus $u_{1} = (i, 1)^{T}$, $u_{2} = (1, -i)^{T}$

\problem{3}
Consider the block upper triangular matrix
$$A = \begin{bmatrix}
A_{11} & A_{12}\\
0 & A_{22}
\end{bmatrix}$$

\subproblem{a}
Suppose $A_{11}u = \lambda u$, but $\lambda$ is not an eigenvalue of $A_{22}$. Find a vector $v$ (in terms of $A_{i,j}$, $u$) such that the vector $\begin{bmatrix} u \\ v\end{bmatrix}$ is an eigenvector of $A$. What is the corresponding eigenvalue?

\solution
$$A \cdot \begin{bmatrix}u \\ v \end{bmatrix} = \begin{bmatrix} A_{11}u + A_{12}v \\ A_{22}v\end{bmatrix} = \begin{bmatrix} \lambda u + A_{12}v \\ A_{22}v\end{bmatrix}$$
Let $v = 0$, then the above equation holds, which indicates that $\begin{bmatrix}u \\ 0 \end{bmatrix}$ is an eigenvector of $A$, with corresponding eigenvalue $\lambda$ which is given by $A_{11}u = \lambda u$.

\subproblem{b}
Suppose $A_{22}v = \lambda v$, but $\lambda$ is not an eigenvalue of $A_{22}$. Find a vector $u$ (in terms of $A_{i,j}$, $v$) such that the vector $\begin{bmatrix} u \\ v\end{bmatrix}$ is an eigenvector of $A$. What is the corresponding eigenvalue?

\solution
$$A \cdot \begin{bmatrix}u \\ v \end{bmatrix} = \begin{bmatrix} A_{11}u + A_{12}v \\ A_{22}v\end{bmatrix} = \begin{bmatrix} A_{11}u + A_{12}v \\ \lambda v\end{bmatrix}$$
Then we know that
$$
\begin{aligned}
&A_{11}u + A_{12}v = \lambda u\\
\Rightarrow &(A_{11} - \lambda I)u = -A_{12}v\\
\Rightarrow &u = -(A_{11} - \lambda I)^{-1}A_{12}v
\end{aligned}
$$
Thus if $A_{11} - \lambda I$ is invertible, $\begin{bmatrix}-(A_{11} - \lambda I)^{-1}A_{12}v \\v \end{bmatrix}$ is an eigenvector of $A$ with corresponding eigenvalue $\lambda$.

\subproblem{c}
Repeat the above assuming $\lambda$ is an eigenvalue for both $A_{11}$ and $A_{22}$. Do any of the above cases fail?

\solution
$$A \cdot \begin{bmatrix}u \\ v \end{bmatrix} = \begin{bmatrix} A_{11}u + A_{12}v \\ A_{22}v\end{bmatrix} = \begin{bmatrix} \lambda u + A_{12}v \\ \lambda v\end{bmatrix}$$
For case (a), if $u$ is given, then set $v$ as 0, $\begin{bmatrix} u \\ 0 \end{bmatrix}$ is still an eigenvector of $A$ with corresponding eigenvalue $\lambda$. Hence case (a) does not fail.

For case (b), the problem leads to the following equation
$$\lambda u = \lambda u + A_{12}v \Rightarrow A_{12}v = 0$$
Thus if $v$ is happened to be in the null space of $A_{12}$, case (b) still holds and $\begin{bmatrix} 0 \\ v \end{bmatrix}$ is the eigenvector with corrfesponding eigenvalue $\lambda$, otherwise case (b) fails.

\end{document}