%!TEX program = xelatex
%# -*- coding: utf-8 -*-
%!TEX encoding = UTF-8 Unicode

\documentclass[12pt,oneside,a4paper]{article}
\usepackage{geometry}
\geometry{verbose,tmargin=2cm,bmargin=2cm,lmargin=2cm,rmargin=2cm}
\usepackage[pdfusetitle,
 bookmarks=true,bookmarksnumbered=true,bookmarksopen=true,bookmarksopenlevel=2,
 breaklinks=false,pdfborder={0 0 1},backref=false,colorlinks=false]
 {hyperref}
\hypersetup{pdfstartview={XYZ null null 1}}
\usepackage{url}
\setcounter{secnumdepth}{2}
\setcounter{tocdepth}{2}
\usepackage{microtype}

\usepackage{amsmath, amsthm, amssymb, amsfonts}

\usepackage{algorithm}
\usepackage{algorithmic}
\renewcommand{\algorithmicrequire}{\textbf{Input:}} 
\renewcommand{\algorithmicensure}{\textbf{Output:}} 

\usepackage[sc]{mathpazo}
\linespread{1.1}
\usepackage[T1]{fontenc}


\usepackage{graphics}
\usepackage{graphicx}
\usepackage[figure]{hypcap}
\usepackage[hypcap]{caption}
\usepackage{tikz}
%\usepackage{grffile} 
%\usepackage{float} 
\usepackage{pdfpages}

\usepackage{multirow}
\usepackage{booktabs}
\usepackage{threeparttable}

%\usepackage[square,numbers,super,comma,sort]{natbib}
%\usepackage[backend=biber, style=nature, sorting=none, isbn=false, url=false, doi=false]{biblatex}
%\addbibresource{ref.bib}
%\usepackage[]{authblk}

\usepackage{verbatim}

\newcommand{\problem}[1]
{
    \clearpage
    \section*{Problem {#1}}
}

\newcommand{\subproblem}[1]
{
    \subsection*{Problem {#1}}
}

\newcommand{\solution}
{
    \vspace{15pt}
    \noindent\ignorespaces\textbf{\large Solution}
}

\usepackage{fancyhdr}
\usepackage{extramarks}
\lhead{\hmwkAuthorName}
\chead{\hmwkTitle}
\rhead{\firstxmark}
\cfoot{\thepage}

\newcommand{\hmwkTitle}{CSCI 5304 HW 1}
\newcommand{\hmwkAuthorName}{Jingxiang Li}

\setlength\headheight{15pt}
\setlength\parindent{0pt}
\setlength{\parskip}{0.3em}

\newcommand{\m}[1]{\texttt{{#1}}}
\newcommand{\trace}{\mathrm{trace}}

\pagestyle{fancy}

\title{\hmwkTitle}
\author{\hmwkAuthorName}
\date{\today}

\begin{document}
\maketitle

\problem{1}
Given two vectors $u,v\in \mathbb{R}^{n}$, and real scalars $\alpha$, $\beta$, let $A = I + \alpha uv^{T}$, $B = I + \beta u v^T$.

\subproblem{a}
If $u, v, \alpha$ are given, find $\beta$ such that $B = A^{-1}$

\solution

$\because B = A^{-1}$\\
$\therefore A\cdot B = I$\\
$\Rightarrow (I + \alpha uv^{T})(I + \beta uv^{T}) = I$\\
$\Rightarrow \alpha uv^{T} + \beta uv^{T} + \alpha\beta uv^{T}uv^{T} = 0$\\
$\Rightarrow \mathrm{trace}(\alpha uv^{T} + \beta uv^{T} + \alpha\beta uv^{T}uv^{T}) = 0$\\
$\Rightarrow \alpha\trace (uv^{T}) + \beta\trace (uv^{T}) + \alpha\beta\trace (uv^{T}uv^{T})$\\
$\Rightarrow \alpha v^{T}u + \beta v^{T}u + \alpha\beta v^{T}uv^{T}u$\\
if $v^{T}u \neq 0$\\
then $\alpha + \beta + \alpha \beta v^{T}u = 0$\\
$\therefore \beta = -\frac{\alpha}{1 + \alpha v^{T}u}$ 
 
\subproblem{b}
For which values of $\alpha$ is $A$ singular, if any? For that particular value of $\alpha$, give a
non-zero vector $x$ in the right nullspace of $A$. Write $x$ in terms of $u$, $v$, $\alpha$.

\solution

$\because A$ is singular\\
$\therefore |I + \alpha uv^{T}| = 0$\\
Following the Sylvester's determinant theorem, we have\\
$|I + \alpha uv^{T}| = 1 + \alpha v^{T}u = 0$\\
$\Rightarrow \alpha = -\frac{1}{v^{T}u}$\\
\\
Set $x = u$
We have $Ax = u - \frac{uv^{T}u}{v^{T}u} = u - u = 0$

\subproblem{c}
Prove of disprove: for any given pair of vector u, v, there always exists a value $\alpha$ such that A singular. To prove, show such an $\alpha$ always exists, giving a formula in terms of u, v. To disprove, give an example of a pair of non-zero vectors u, v for which no such $\alpha$ exists. In the latter case, what general property do $u$, $v$ satisfy to prevent the existence of $\alpha$? You can illustrate your answer with a $2 \times 2$ example.

\solution

\textbf{Disprove:} When $v^{T}u = 0$, no such $\alpha$ exists. \\
Because when $v^{T}u = 0$, $|I + \alpha uv^{T}| = 1 + \alpha v^{T}u = 1 > 0$;\\
For example, let $u = (1, 1)^{T}$, $v = (1, -1)^{T}$, Then $A$ is always non-singular.

\subproblem{d}
Give a value of $\alpha$ (in terms of $u$, $v$) such that $A^2 = A$ (i.e., $A$ is a projector).

\solution

$\because A^2 = A$\\
$\therefore (I + \alpha uv^{T})^2 = I + \alpha uv^{T}$\\
$\Rightarrow I + \alpha uv^{T} + \alpha uv^{T} + \alpha^2 uv^{T}uv^{T} = I + \alpha uv^{T}$\\
$\Rightarrow \alpha uv^{T} + \alpha^2 uv^{T}uv^{T} = 0$\\
Thus $\alpha = 0$ is one solution.\\
Otherwise, $\trace(uv^{T}) + \alpha \trace(uv^{T}uv^{T}) = 0$\\
$\Rightarrow v^{T}u + \alpha v^{T}uv^{T}u = 0$\\
if $v^{T}u \neq 0$\\
$\alpha = -\frac{1}{v^{T}u}$\\
So that we have $\alpha = 0$ or $\alpha = -\frac{1}{v^{T}u}$.

\problem{2}
Let $f_{p}(v) = \max_{||u||_{p} = 1}|u^{T}v|$, where $||v||_{p}$ denotes the p-norm.

\subproblem{a}
Prove or disprove: $f_{p}$ is a vector norm. (check each property, or show one is violated).

\solution

\emph{Proof.}
\begin{enumerate}
    \item $f_{p}(v) = \max_{||u||_{p} = 1}|u^{T}v| > 0$
    \item $\forall \in \mathbb{R},~f_{p}(av) = \max_{||u||_{p} = 1}|u^{T}av| = \max_{||u||_{p} = 1}a|u^{T}v| = af_{p}(v)$
    \item $f_{p}(x + y) = \max_{||u_{1}||_{p} = 1}|u^{T} (x + y)| \leq |u_{1}^{T}x| + |u_{1}^{T}y| \leq \max_{||u_{x}||_{p} = 1}|u_{x}^{T}x| + \max_{||u_{y}||_{p} = 1}|u_{y}^{T}y| = f_{p}(x) + f_{p}(y)$
\end{enumerate}

Therefore $f_{p}$ is a vector norm.

\subproblem{b}
Give a formula for $f_{p}$ for $p = 1,2$. Hint, the answers can be written in terms of $||\cdot||_{2}$, $||\cdot||_{\infty}$. For $p = 2$, use the Cauchy-Schwartz inequality.

\solution

$f_{1}(v) = \max_{||u||_{1} = 1}|u^{T}v| = ||v||_{\infty}$\\
$f_{2}(v) = \max_{||u||_{2} = 1}|u^{T}v| \leq ||u||_{2}\cdot ||v||_{2} = ||v||_{2}$\\

\problem{3}
Define the inner product among square matrices by $<A, B> = \trace(A^{T}B)$, where $A$, $B$ are $n \times n$ matrices.

\subproblem{a}
What is the norm induced by this inner product: $||A|^{2}=<A, A>$? Answer this question for the general case for any $A$.

\solution

$||A|| = \sqrt{\trace(A^{T}A)}$

\subproblem{b}
Now answer the remaining questions below using this specific matrix:
$$A = \begin{pmatrix}
  6 & -2 & 1\\
  7 & -7 & 3\\
  -4 & 5 & -2
 \end{pmatrix}$$
For this specific matrix $A$, what is the value of $<A, A>$ and the corresponding induced norm $||A||^{2}=<A, A>$ from part (a)?

\solution 

$<A, A> = \trace(A^{T}A) = 193$\\
$||A|| = \sqrt{<A, A>} = \sqrt{193}$

\subproblem{c}
What is the $p$ norm of $A$, for $p = 1$? Find a vector $x$ s.t. $||A||_{p} = ||Ax||_{p}$.

\solution

Assume the dimension of $A$ is $n \times m$\\
$||A||_{1} = \max_{||x||_{1} = 1}||Ax||_{1} = 1_{1\times n} \cdot A \cdot x = \max_{||x||_{1} = 1} (s_{1},\dots,s_{m}) \times x$\\
Where $s_{i}$ is the $i$th column sum of absolute value of matrix A. Since $||x||_{1} = 1$, we have\\
$||A||_{1} = \max_{i}{s_{i}}$ = 17\\
$x = (1, 0, 0)^{T}$

\subproblem{d}
Repeat the above for $ p = 2$. Use Matlab and write the result to 4 decimal places. Show your Matlab commands.

\solution
\begin{verbatim}
format short;
A = [6, -2, 1; 7, -7, 3;-4, 5, -2];
[EVECT,EVAL]=eig(A' * A);
sqrt(max(abs(diag(EVAL))))
> 
13.595
x = EVECT(:,3)
> 
0.72496
-0.63330
0.27086

\end{verbatim}
$||A||_{2} = 13.595$\\
$x = (0.7250, -0.6333, 0.2709)^{T}$

\subproblem{e}
Use Matlab to help solve this problem: Find a vector $x$ achieving the minimum in $\min_{||x||_{p} = 1}||Ax||_{p}$. Do this for $p = 1, 2$.

\solution

For $p = 1$, similar to the maximum problem, the minimum result should be the smallest value of sum of absolute column values.
\begin{verbatim}
min(sum(abs(A),1))
> 6
\end{verbatim}
$\min_{||x||_{1} = 1}||Ax||_{1} = 6$\\
$x = (0, 0, 1)^{T}$
\\\\\\
For $p = 2$, similar to the maximum problem, the minimum result should be the smallest square root of absolute eigenvalue.
\begin{verbatim}
[EVECT,EVAL]=eig(A' * A);
sqrt(min(abs(diag(EVAL))))
> 
0.077258

x = EVECT(:,1)
>
-0.040332
0.353531
0.934553
\end{verbatim}
$\min_{||x||_{p} = 2}||Ax||_{2} = 0.0773$\\
$x = (-0.0403, 0.3535, 0.9346)^T$
\end{document}


